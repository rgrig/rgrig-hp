\documentclass[a4paper,12pt]{article}

\usepackage[T1]{fontenc}
\usepackage[scaled]{helvet}
\usepackage{hyperref}
\usepackage{microtype}
\usepackage{xcolor}
\usepackage{xspace}

\renewcommand*\familydefault{\sfdefault}

\setlength{\hoffset}{-0.54cm}
\setlength{\voffset}{-0.54cm}
\setlength{\oddsidemargin}{0pt}
\setlength{\evensidemargin}{0pt}
\setlength{\topmargin}{0pt}
\setlength{\headheight}{0pt} % no header
\setlength{\headsep}{0pt}
\setlength{\textheight}{24.75cm}
\setlength{\textwidth}{17cm}
\setlength{\marginparsep}{0pt} % no margin
\setlength{\marginparwidth}{0pt}
\setlength{\marginparpush}{0pt}
\setlength{\footskip}{27pt}

\showboxdepth=10
\showboxbreadth=100


\definecolor{darkblue}{rgb}{0,0,0.4}
\hypersetup{colorlinks,linkcolor=darkblue,citecolor=darkblue,urlcolor=darkblue}
\hypersetup{
  pdfauthor={Radu Grigore},
  pdftitle={Curriculum Vitae of Radu Grigore}}

\newcommand{\escjava}{\href{http://en.wikipedia.org/wiki/ESC/Java}{ESC\slash Java}\xspace}
\newcommand{\nobug}{\href{http://www.nobug.com/}{NoBug Consulting}}
\newcommand{\psl}{\href{http://www.eda-stds.org/ieee-1850/}{PSL}\xspace}

%TODO: transform spaces into +. In the meantime use only simple words.
\newcommand{\google}[1]
{\href{http://www.google.com/search?q=#1\&btnI=I'm+feeling+lucky}{#1}}
\renewcommand{\today}{\ifcase\month\or
  January\or February\or March\or April\or May\or June\or
  July\or August\or September\or October\or November\or December\fi
  \space\number\year}
\def\,#1{{\lineskiplimit=0pt \oalign{\relax#1\crcr\hidewidth,\hidewidth}}}

\newcommand{\entry}[2]{%
  \hbox to \textwidth{%
    \hbox to 2.5cm{\hfil\parfillskip=0pt #1}%
    \hss%
    \vtop{\hsize=14.2cm\raggedright\noindent#2}%
  }%
  \medskip%
}
\newcommand{\gb}{\penalty-1} % good break

\overfullrule=5pt

\begin{document}
\centerline{\Huge\bfseries Radu Grigore}
\vskip .5cm plus.5cm minus.3cm
\centerline{\today}
\vskip 1cm plus.5cm minus.5cm

\entry{\emph{Email}:}{\href{mailto:radugrigore@gmail.com}
  {\nolinkurl{radugrigore@gmail.com}}}
\entry{\emph{Homepage}:}{\url{http://rgrig.appspot.com/}}
\entry{\emph{Blog}:}{\url{http://rgrig.blogspot.com/}}

\section*{Education}

\entry{2013--now}
  {Postdoctoral Research Assistant, \gb
  Computer Science, \gb
  University of Oxford.}
\entry{2010--2013}
  {Postdoctoral Research Assistant, \gb
  Theory Group, \gb
  EECS, \gb
  Queen Mary, University of London.}
\entry{2005--2010}
  {PhD, \gb
  CS, \gb
  \emph{The Design and Algorithms of a Verification Condition Generator}, \gb
  University College Dublin.}
\entry{1998--2003}
  {BSc, \gb
  EE, \gb
  Politehnica University of Bucharest. \gb
  My average grade is 9.75 out of~10. \gb
  My diploma dissertation is on \emph{Traffic Models for Data Networks}.}
\entry{2001}
  {Internship in the Darmstadt Institute for Microelectronics, \gb
  funded by a \google{DAAD} scholarship.}
\entry{1994--1998}
  {``Gheorghe \,Sincai'' highschool, Bucure\,sti}


\section*{Industry}

\entry{2008}
  {I experimented with using kd-trees and R-trees to add 
  orthogonal range queries on top of \google{BigTable} as a Software 
  Engineer Intern in \google{Google}.}
\entry{2004--2005}
  {I helped design and develop a translator between two
  languages used in hardware verification. The client was
  \href{http://www.synopsys.com/}{Synopsys} and I was working in a
  team of ten within \nobug.}
\entry{2003--2004}
  {I designed and developed a PSL frontend for 
  the model checker \google{RuleBase} while working for \nobug.}
\entry{2003}
  {I helped design and develop a translator from \psl
  to finite automata with counters, whose implementation is
  generated in a variety of procedural languages. I worked
  in a team of two within \nobug.}

\pagebreak
\section*{Teaching}

\entry{2011}
  {\emph{Specification and Verification}
    (QMUL): I led problem solving sessions.}
\entry{2011}
  {\emph{Probabilities and Matrices}
    (QMUL): I demonstrated during practicals.}
\entry{2011}
  {\emph{Algorithms and Data Structures in an Object-Oriented Framework}
    (QMUL): I demonstrated during practicals.}
\entry{2009}
  {\emph{Unix Programming} (UCD): I designed the course and
  I delivered the lectures.}
\entry{2007--2008}
  {Coach of the team that represented UCD (and Ireland) in 
  \href{http://icpc.baylor.edu/}{ACM ICPC}.}
\entry{2008}
  {\emph{Foundations of Computing} (UCD): I demonstrated during practicals.}
\entry{2008}
  {\emph{Operating Systems} (UCD): I designed the course and
  I delivered the lectures.}
\entry{2006--2007}
  {\emph{Algorithmic Problem Solving} (UCD): I demonstrated during practicals.}
\entry{2006}
  {\emph{Data Structures and Algorithms} (UCD): I demonstrated during practicals.}
\entry{2004}
  {\emph{Telecommunications Software} (PUB): I demonstrated during practicals.}
\entry{2003}
  {\emph{Data Networks} (PUB): I demonstrated during practicals.}

\section*{Selected Publications}
\entry{2015}
  {Radu Grigore,  Stefan Kiefer.
  \newblock \emph{Tree Buffers}.
  \newblock CAV}
\entry{2014}
  {Xin Zhang,  Ravi Mangal,  Radu Grigore,  Mayur Naik,  Hongseok Yang.
  \newblock \emph{On Abstraction Refinement for Program Analyses in Datalog}.
  \newblock PLDI ({\bf distinguished paper award})}
\entry{2013}
  {Mikol\'a\v{s} Janota,  Radu Grigore,  Joao Marques-Silva.
  \newblock \emph{On QBF Proofs and Preprocessing}.
  \newblock LPAR}
\entry{2013}
  {Radu Grigore,  Dino Distefano,  Rasmus Lerchedahl Petersen,  Nikos Tzevelekos.
  \newblock \emph{Runtime Verification Based on Register Automata}.
  \newblock TACAS}
\entry{2013}
  {Nikos Tzevelekos,  Radu Grigore.
  \newblock \emph{History-Register Automata}.
  \newblock FoSSaCS}
\entry{2012}
  {Thorsten Bormer, 
  Marc Brockschmidt, 
  Dino Distefano, 
  Gidon Ernst, 
  Jean-Christophe Filli\^atre, 
  Radu Grigore, 
  Marieke Huisman, 
  Vladimir Klebanov, 
  Claude March\'e, 
  Rosemary Monahan, 
  Wojciech Mostowski, 
  Nadia Polikarpova, 
  Christoph Scheben, 
  Gerhard Schellhorn, 
  Bogdan Tofan, 
  Julian Tschannen, 
  Mattias Ulbrich.
  \newblock \emph{The COST IC0701 verification competition 2011}.
  \newblock FoVeOOS}
\entry{2011}
  {Matko Botin\v{c}an,  Dino Distefano,  Mike Dodds,  Radu Grigore, 
  Daiva Naud\v{z}i\=unien\.e,  Matthew J. Parkinson.
  \newblock \emph{coreStar: The Core of jStar}.
  \newblock BOOGIE}
\entry{2010}
  {Mikol\'a\v{s} Janota,  Goetz Boetterweck,  Radu Grigore, 
  Joao Marques-Silva. 
  \newblock \emph{How to Complete a Configuration Process?}
  \newblock SOFSEM}
\entry{2009}
  {Radu Grigore,  Julien Charles,  Fintan Fairmichael,  Joseph Kiniry.
  \newblock \emph{Strongest Postcondition of Unstructured Programs}.
  \newblock FTfJP}
\entry{2007}
  {Mikol\'a\v{s} Janota,  Radu Grigore,  Micha{\l} Moskal.
  \newblock \emph{Reachability Analysis for Annotated Code}.
  \newblock SAVCBS}
\entry{2005}
  {Valentin \,Stefan Gheorghi\,t\u{a}, Radu Grigore.
  \newblock \emph{Constructing Checkers from PSL Properties}.
  \newblock CSCS}

\section*{Open Source Contributions}

\entry{2011$\ldots$}
  {\href{http://rgrig.github.com/topl/}{TOPL} is a
  runtime verifier for temporal properties of Java programs.}
\entry{2010$\ldots$}
  {I help maintain the verifier
  \href{https://github.com/seplogic/corestar}{coreStar} and
  its frontend
  \href{https://github.com/seplogic/jstar}{jStar}.}
\entry{2007$\ldots$}
  {\href{http://code.google.com/p/freeboogie/}{FreeBoogie} is a 
  program verifier backend.}
\entry{2007--2010}
  {AstGen generates code from abstract grammars. It is part
  of \href{http://code.google.com/p/freeboogie/}{FreeBoogie}.}
\entry{2008--2010}
  {\href{http://clops.sf.net/}{CLOPS} is a code generator 
  for command line parsing.}
\entry{2005--2009}
  {I maintained \escjava and I implemented its reachability analysis.}
\entry{2008}
  {Part of my work in Google is included in \google{UzayGezen}.}
\entry{2005}
  {\href{http://cfind.sf.net/}{CFind} indexes and searches
  a hard-drive. I might improve it sometime.}


%\section*{Personal Statement}
%
%I want to build bridges in computing science. The largest rift
%is between theoreticians and practitioners, but it is certainly
%not the only one. The research community is partitioned into
%small enclaves, each with its own language and its own norms.
%Such grouping is desirable as it is impossible for one person to
%understand all the recent research results in computing science.
%Individuals cope with information overload by spending more time
%to digest information from trusted and familiar sources than from
%other sources. In such an environment it is not too surprising
%that sometimes surprising links are found between the results of
%different communities. I want to spend most of my time searching
%for connections between the results of various communities and
%translating from one language into another. The task is hard
%because it involves being abreast with more than one field.
%
%The areas I enjoy most are \emph{program verification},
%\emph{programming languages}, and \emph{algorithms}. The goal of
%the program verification community is to find good methods to
%reason formally about the correctness of programs; the goal of
%the programming language community is to devise expressive and
%robust ways to capture algorithms; the goal of the algorithms
%community is to find efficient ways to solve repetitive problems.
%I want to continue searching for links between these areas and,
%in the process, develop tools that practitioners can use to write
%better software.

\end{document}

